% ┌---------------------------------------------┐
% | Titulo:	03-Formato de texto, salto de linea |
% | Autor:	Jaimito                             |
% | Fecha: 	15/12/2016                          |
% └---------------------------------------------┘
\documentclass[10pt,a4paper]{book}
\usepackage[utf8]{inputenc}
\usepackage{amsmath}
\usepackage{amsfonts}
\usepackage{amssymb}
\begin{document}
% ┌-----------------------------------------┐
% | Prueba del salto de línea y comentarios |
% └-----------------------------------------┘
\chapter{Saltos de línea}
 La primera línea del documento \\
 La segunda línea del documento
\chapter{Saltos en el documento}

	\section{Saltos verticales}
	 Los saltos verciales empezando por el pequeño \smallskip \\
	 Continuamos con el mediano \medskip \\
	 Acabamos con el grande \bigskip \\
	 A continuación un salto personalizado \vspace{1cm} \\
	 Y vemos los saltos 
	
	\section{Saltos horizontales}
	 Los saltos horizontales de una línea a \hspace{1.2in} 1.2 pulgadas
\chapter{centrado}
 El texto centrado queda así:
	\begin{center}
	 Con 10 cañones por banda, \\
	 viento en popa a toda vela \\
	 no corta el mar sino vuela \\
 	 un velero bergatín \vspace{2cm}\\ 
 	 
 	 "Canción del pirata" \\
 	 Espronceda
		
	\end{center}
	 Ahora un texto centrado sin más
	\begin{center}
	 En la lona gime el viento y alza en blanco movimiento olas de plata 
	 y azul. Navega velero mío sin temor, que ni tormenta ni enemigo ni 
	 bonanza, tu rumbo a torcer alcanza y a doblegar tu valor.
 	 "Canción del pirata" \\
 	 Espronceda
		
	\end{center}
	\section{Tipos de letra}
	 El tipo de letra {
	 	\em en cursiva
	 } y ahora normal. Lo siguiente es probar 
	 en seriff {
	 	\sf letras en seriff {
	 		\em solo en cursiva {
	 			\bf negrita y cursiva	 		
	 		}
	 	}
	 } y ahora normal
\end{document}